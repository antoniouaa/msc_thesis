The ability to predict changes in stock prices is extremely important to the financial world as it influences trading strategies and reduces risks in the market. Forecasting has long been a problem for the business and technology communities and has seen little advances until quite recently, with the advent of neural networks and deep learning.

Before artificial neural networks, the finance world used other methods to model the time series that arose from the continuous updating of stock prices. Models like the autoregressive integrated moving average model (ARIMA) and the generalised autoregressive conditional heteroscedasticity model (GARCH) have become key econometric methods for forecasting time series and are still widely used in finance.

The focus of this project is to provide a comparison between the traditional methods for time series forecasting mainly the ARIMA model, and simple implementations of artificial neural networks, in the context of financial time series prediction.

Forecasting stock prices with moving averages belongs to the technical analysis category of financial analysis. \citeauthor{tech_analysis} (\citeyear{tech_analysis}) define \textit{technical analysis} as the study of prices in freely traded markets with the intent of making profitable trading or investment decisions,\citesuper{tech_analysis} hence the models will only use daily prices for the selected stocks, ignoring company financial data for both forecasting methods, ANNs and autoregressive models.

\section{Problem statement}
The Efficient Market Hypothesis states that stock prices reflect all available information and respond to any changes in that information through price changes. An implication of this relation between available information and the market is that consistently predicting a change in an asset price is hard. Followers of this hypothesis believe that the market instantly responds to new information made available and so technical analysis of assets is moot. The Efficient Market Hypothesis finds its roots in the works of \citeauthor{bachelier} (\citeyear{bachelier}), \citeauthor{mandelbrot} (\citeyear{mandelbrot}) and \citeauthor{samuelson} (\citeyear{samuelson}) but more closely associated to \citeauthor{fama_efficient_market} whose published review paper, \emph{\citetitle{fama_efficient_market}}, proposed the market efficiency types and the \emph{joint hypothesis problem} that introduces certain requirements for testing market efficiency.\citesuper{bachelier}\citesuper{mandelbrot}\citesuper{samuelson}\citesuper{fama_efficient_market} 

Of course, the EMH being near or even completely untestable means that a lot of research has been done to provide insight to the problem of predicting the stock market, both for the hypethesis and against. \citeauthor{grossman-stiglitz} (\citeyear{grossman-stiglitz}) propose a model


