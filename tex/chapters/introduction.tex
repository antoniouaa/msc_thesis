The ability to predict changes in stock prices is extremely important to the financial world as it influences trading strategies and reduces risks in the market. Forecasting has long been a problem for the business and technology communities and has seen little advances until quite recently, with the advent of neural networks and deep learning.

Before artificial neural networks, the finance world used other methods to model the time series that arose from the continuous updating of stock prices. Models like the autoregressive integrated moving average model (ARIMA) and the generalised autoregressive conditional heteroscedasticity model (GARCH) are key econometric methods for forecasting time series and are still widely used in finance.

The focus of this project is to provide a comparison between the traditional methods for time series forecasting mainly the ARIMA model, and simple implementations of artificial neural networks, in the context of financial time series prediction.

Forecasting stock prices with moving averages belongs to the technical analysis category of financial analysis. \citeauthor{tech_analysis} define \textit{technical analysis} as the study of prices in freely traded markets with the intent of making profitable trading or investment decisions\cite{tech_analysis}, hence the models will only use daily prices for the selected stocks, ignoring company financial data.



