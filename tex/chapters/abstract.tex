\begin{abstract}
Properly pricing options is a challenging task. The traditional method of the Black-Scholes model makes a number of critical assumptions that do not hold up to reality and do not deliver results with as much desired accuracy. This paper proposes the use of Generative Adversarial Networks (GANs) in pricing European options. GANs can quickly learn to generate data distributions with realistic properties in a data-driven approach. In this paper we (i) explore methods of tuning the model's parameters, (ii) test the architecture on a curated dataset of European instruments and  (iii) evaluate the performance of the networks, comparing the results to already established pricing methods, like the Black-Scholes model, as well as real derivative data.
\\
\headrule
{\it Keywords: Deep Learning, Generative Adversarial Networks, Options pricing, Neural Networks}
\end{abstract}